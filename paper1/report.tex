\documentclass[sigconf]{acmart}

\usepackage{hyperref}

\usepackage{endfloat}
\renewcommand{\efloatseparator}{\mbox{}} % no new page between figures

\usepackage{booktabs} % For formal tables

\settopmatter{printacmref=false} % Removes citation information below abstract
\renewcommand\footnotetextcopyrightpermission[1]{} % removes footnote with conference information in first column
\pagestyle{plain} % removes running headers

\begin{document}
\title{Big Data Applications in Maternal Death During Childbirth}


\author{Elena Kirzhner}
\affiliation{%
  \institution{Indiana University}
  \city{Bloomington, IN 47408} 
  \country{USA}}
\email{ekirzhne@iu.edu}


% The default list of authors is too long for headers}
\renewcommand{\shortauthors}{B. Trovato et al.}


\begin{abstract}
With the major growth of big data and applications to collect, analyze and store unstructured and structured data it makes it possible to impact and analyze patterns in maternal death during childbirth to predict who's susceptible to fatality and what can be done to prevent it.
\end{abstract}

\keywords{i523, hid320, Big Data Applications and Analytics, Data Science, Maternal Mortality}


\maketitle

\section{Introduction}

Maternity death is rising for unclear reasons in United States. USA is the only developed nation where that rate is increasing. American women are more likely to die from childbirth than women in any other high developed country. Based on research and analysis by the Center for Disease Control and Prevention \cite{bacak2006state}, maternal death doubled from 2000-2014 and more than half of such incidents could be prevented with the current medical technology. Most of the cases were result of medical error and unprepared hospitals. Doctor’s ability to protect the health of mothers in childbirth is a basic measure of a society’s development.Yet every year in the United States 700 to 900 women die from pregnancy or childbirth-related causes, and some 65,000 nearly die by many measures, the worst record in the developed world \cite{world2012trends} and \cite{amnesty2010deadly}.
We have ability to prevent it and predict with monitoring the cases and usage of the Big Data and Analytics. 

\section{Background}

Statistical research for 2010 put American in the 50Th place; the lowest of all developed nations for maternal death during childbirth (Fig. 1). From 1990 to 2014 pregnancy related death increased by 1.7  percent while worldwide that rate decreased by 1.3 percent (Fig. 2). Thus, proper calculation shows that maternity mortality rate practically doubled in the last decade. Women giving birth in Asia have lower risk to die than those giving birth in United States \cite{world2012trends}. Currently, researches are inconclusive, as to why the rate is rising in USA. Multiple variables are being taken into account, such as race, age and economic status \cite{creanga2012race}.

\subsection{Definition}

According to the National Center for Health Statistics, Pregnancy Mortality Surveillance System and the International Classification of Disease, to properly analyze data, causes of death during child birth were categorized and defined \cite{callaghan2012overview} as follows:

1. Pregnancy related death - death during the first 42 days after giving birth that is directly related to pregnancy and health care. Not  
   related to any accidents outside of the pregnancy.

2. Maternal fatality ratio - death caused by pregnancy for every 100,000 pregnancy occurrences.

\subsection{Monitoring}

The National Center for Health Statistics requires all states on annual basis to provide death certificates with causes of maternal death. This data is analyzed and compared against international statistics \cite{hoyert2007maternal}  and \cite{creanga2014maternal}.

Additionally, Pregnancy Mortality Surveillance System was implemented in 1896, because of limited pregnancy death related records  \cite{horon2011effectiveness}. This system was created to record and analyze all pregnancy related deaths.  Every year, this group sends a request to all 50 states to provide death certificate copies for those who died during childbirth and pregnancy. This data is stored and further analyzed by trained doctors and data scientists. That group coined new term ''pregnancy-related mortality'' \cite{callaghan2012overview}. This information is being released in Center for Disease Control and Prevention Morbidity and Mortality Weekly reports and their website \cite{neggers2016trends}. Death related to pregnancy from 1998-2010 were published in Obstetrics and Gynecology journal \cite{schulz1994assessing}. Furthermore, since launching the program, monitoring and analyzing the data, rate has dramatically increased from 7.2 deaths per 100,000 births in 1987 to 17.8 deaths per 100,000 births in 2011 \cite{neggers2016trends} and (Fig. 3).


\section{Big Data Usage And How It Can Help}

The causes of these death are not yet identified since only limited amount of data was analyzed \cite{creanga2012race}. Big data tools help to understand and organize pregnancy related deaths and causes. Also it helps to collect and identify risks by race ethnicity, economical status and age. Further examination of structured and unstructured data could help with preventing causes of pregnancy related death.

A similar study was done on October 8, 2016 by  journal The Lancet, that called ''Global, regional, and national levels of maternal mortality, 1990–2015: a systematic analysis for the Global Burden of Disease Study 2015'' \cite{kassebaum2016global}. They used a standardized process to identify, extract and process all relevant data sources. Standardized algorithms were implemented to adjust for age-specific, year-specific, and geography-specific patterns of incompleteness, as well as patterns of miss-classification of deaths \cite{mcginnis2013best}.

Internet Of Things could be used to  monitor patients and their pregnancy risks  such as diabetes level or blood pressure. It could also track prescribed medicine, it’s especially useful for patients without health insurances \cite{kassebaum2016global}. 

Predictive analytic should be used, women’s information could be shared between doctors and hospitals to be diagnosed in advance, improving number of healthy pregnancies. By being able to analyze big data, pregnancy risks will be predicted and provide women with safety and better pregnancy outcomes. The more analyzed data we have, the sooner it will reduce the mortality rates and it will be easier to diagnose each case. Special kits with appropriate medicine could be supplied to each hospital for individual patient.

The data could be put into Hadoop to make a more scale-able analysis with that. Possibility to get more accurate causes and reasons. Hadoop system is an open source software for distributed storage of large datasets on computer clusters and visualization. 


\section{Conclusion}

Pregnancy-related mortality findings should be recorded and cross analyzed. It provides a better view, results clarification and better health management. Additionally it will decrease same errors and doctors faults and prevent maternity death. All these years, there was not enough information that was structured for deeper analysis. Big Data getting bigger daily, this information is everywhere including emails, doctor’s notes, lab tests, medications. Different platforms such as Hadoop can keep and analyze huge mass of information. Doctors and medical staff could use that information to improve pregnant mother’s health for better outcomes and prevent death. In addition, it will lower medical costs.



\appendix

%Appendix A
\section{Headings in Appendices}

The rules about hierarchical headings discussed above for the body of
the article are different in the appendices.  In the \textbf{appendix}
environment, the command \textbf{section} is used to indicate the
start of each Appendix, with alphabetic order designation (i.e., the
first is A, the second B, etc.) and a title (if you include one).  So,
if you need hierarchical structure \textit{within} an Appendix, start
with \textbf{subsection} as the highest level. Here is an outline of
the body of this document in Appendix-appropriate form:

\subsection{Introduction}
\subsection{Background}
\subsubsection{Definitions}
\subsubsection{Monitoring}
\subsection{Big Data Usage And How It Can Help}
\subsection{Conclusion}
\subsection{References}

Generated by bibtex from your \texttt{.bib} file.  Run latex, then
bibtex, then latex twice (to resolve references) to create the
\texttt{.bbl} file.  Insert that \texttt{.bbl} file into the
\texttt{.tex} source file and comment out the command
\texttt{{\char'134}thebibliography}.

% This next section command marks the start of
% Appendix B, and does not continue the present hierarchy

\section{More Help for the Hardy}

Of course, reading the source code is always useful.  The file
\path{acmart.pdf} contains both the user guide and the commented code.

\begin{acks}

  The authors would like to thank Dr. Yuhua Li for providing the
  matlab code of the \textit{BEPS} method.

  The authors would also like to thank the anonymous referees for
  their valuable comments and helpful suggestions. The work is
  supported by the \grantsponsor{GS501100001809}{National Natural
    Science Foundation of
    China}{http://dx.doi.org/10.13039/501100001809} under Grant
  No.:~\grantnum{GS501100001809}{61273304}
  and~\grantnum[http://www.nnsf.cn/youngscientsts]{GS501100001809}{Young
    Scientsts' Support Program}.

\end{acks}

\bibliographystyle{ACM-Reference-Format}
\bibliography{report} 

\end{document}
